\section {Computational Methods}

Classical molecular dynamics (MD) simulations were performed using the Amber 11 suite of simulation programs.\cite{XXX} The simulated system of water and malonic acid was initialized by creating a cubic unit cell with side lengths of 30\angs. The unit cell was then randomly packed with 900 water molecules, and 4 malonic acid molecules using the PACKMOL program created to simplify construction of MD starting configurations.\cite{Packmol} This resulted in a malonic acid concentration of 2.5 M, which was purposefully set to be near that of the experimental VSFG conditions.

The initial system was energy minimized by a geometry optimization method. The z-axis of the system was then expanded to 100\angs~creating a large vacuum region adjacent to the aqueous box. Periodic boundary conditions were then employed resulting in an infinite slab configuration\cite{XX} with two aqueous-vacuum interfaces. This configuration was then evolved through simulation for 500 ps to further equilibrate the system. The system was then evolved for data-collection for a further 50 ns trajectory, recording atomic coordinates every 100 fs for a total of 500,000 data points.

The simulations were performed using a timestep of 1 fs. Fully polarizable models were used for both the water and malonic acid molecules. Water was simulated using the POL3 model,\cite{XX} and the malonic acids were constructed using a fully atomistic model based on the Amber FF02EP force field. The system temperature was set at 298K, and Langevin dynamics were used to propoagate dynamics via a leapfrog integrator. The particle mesh ewald technique was used for calculating long-range electrostatic interactions, with a force cutoff set to 10\angs. Waters were held rigid by means of the SHAKE algorithm to increase computational throughput and speed of data collection.

In all following analyses, the results obtained for molecular orientation are averaged between both of the water slab surfaces. The distance to each aqueous surface was determined for every malonic acid at each timestep. The closer surface was always used to analyze acid orientation, and the reference axis was always set to point from the aqueous bulk outwards towards the gas phase, normal to the plane of the water surface. Angle calculations were made with respect to this surface-normal reference axis.
